% Created 2024-04-29 Mon 11:47
% Intended LaTeX compiler: pdflatex
\documentclass[11pt]{article}
\usepackage[utf8]{inputenc}
\usepackage[T1]{fontenc}
\usepackage{graphicx}
\usepackage{longtable}
\usepackage{wrapfig}
\usepackage{rotating}
\usepackage[normalem]{ulem}
\usepackage{amsmath}
\usepackage{amssymb}
\usepackage{capt-of}
\usepackage{hyperref}
\author{sunny}
\date{\today}
\title{}
\hypersetup{
 pdfauthor={sunny},
 pdftitle={},
 pdfkeywords={},
 pdfsubject={},
 pdfcreator={Emacs 29.3 (Org mode 9.6.24)}, 
 pdflang={English}}
\begin{document}

\tableofcontents

\section{Programming Assignment}
\label{sec:org848fb0e}

\subsection{B-TECH 2nd Semester, Roll NO: CSE-23 41}
\label{sec:org7c60f79}

\begin{enumerate}
\item A number is called an Armstrong number if the sum of the cubes of the digits of the
\end{enumerate}
number is equal to the number. For example 153 = 1\textsuperscript{3} + 5\textsuperscript{3} + 3\textsuperscript{3}. Write a C program
that asks the user to enter a number and returns if it is Armstrong or not. 

\begin{verbatim}
#include <math.h>
#include <stdio.h>

int main() {
  int numberCopy, number, remainder, result = 0, n = 0;
  printf("Enter the number: ");
  scanf("%d", &number);
  numberCopy = number;


  for (; numberCopy != 0; numberCopy /= 10, ++n);

  numberCopy = number;

  while (numberCopy != 0) {
    remainder = numberCopy % 10;
    result += pow(remainder, n);
    numberCopy /= 10;
  }

  if (result == number) {
    printf("Armstrong number\n");
  } else
    printf(" Not an Armstrong number\n");
  return 0;
}
\end{verbatim}


\begin{enumerate}
\item Write a C program that calculates the HCF and LCM of two numbers.

\begin{verbatim}
#include <stdio.h>

int main() {
  int a, b;
  printf("Enter the two numbers: ");
  scanf("%d%d", &a, &b);

  int large = (a > b) ? a : b;
  int hcf = 1;

  // find the HCF
  for (int i = 1; i <= large; i++) {
    if (a % i == 0 && b % i == 0) {
      hcf = i;
    }
  }
  printf("The HCF of %d and %d is %d \n", a, b, hcf);

  // find the LCM
  int lcm = (a * b) / hcf;
  printf("The LCM of %d and %d is %d \n", a, b, lcm);
  return 0;
}
\end{verbatim}

\item Write a C program to display and find the sum of the series 1+11+111+\ldots{}.111 upto n. For eg. if n=4, the series is : 1+11+111+1111. Take the value of 'n' as input from the user.

\begin{verbatim}
#include <math.h>
#include <stdio.h>

int main() {
  int a = 0, result = 0, newValue = 0;
  printf("Enter the value of n: ");
  scanf("%d", &a);
  for (int i = 0; i < a; i++) {
    newValue = newValue + pow(10, i);
    // printf("%d + ", newValue);
    result += newValue;
  }
  // printf("\n");
  printf("answer = %d \n", result);
  return 0;
}
\end{verbatim}

\item Amicable numbers are found in pairs. A given pair of numbers is Amicable if the sum of
\end{enumerate}
the proper divisors (not including itself) of one number is equal to the other number and
vice – versa. For example 220 \& 284 are amicable numbers
First we find the proper divisors of 220:
220:1, 2, 4, 5, 10, 11, 20, 22, 44, 55, 110
1+ 2 + 4 + 5 + 10 + 11 + 20 + 22 + 44 + 55 + 110 = 284
Now, 284: 1, 2, 4, 71, 142
1 + 2 + 4 + 71 + 142 = 220
Write a C program to check that the input pair of numbers is amicable.

\begin{verbatim}
#include <stdio.h>

int main() {
  int a = 0, b = 0, divisors = 0, newValue = 0;
  printf("Enter the two numbers to find the amicable of(use spaces): ");
  scanf("%d%d", &a, &b);
  for (int i = 1; i <= a; i++) {
    if (a % i == 0 && a != i) {
      divisors = i;
      newValue += divisors;
    }
  }
  if (newValue == b) {
    printf("the numbers are amicable");
  } else {
    printf("the numbers are not amicable");
  }
  return 0;
}
\end{verbatim}

\begin{enumerate}
\item Write a C program to find the reverse of an integer number. 

\begin{verbatim}
#include <stdio.h>

int main() {
  int a = 0, temp = 0;
  printf("Enter the integer at number you want to find the reverse of: ");
  scanf("%d", &a);
  if (a >= 0) {
    // while (a != 0) {
    for (int i = 0; i < a;) {
      int b = a % 10;
      temp = temp * 10 + b;
      a = a / 10;
    }
    printf("the reversed number is: %d", temp);
  } else {
    printf("The number is not an integer, correct the input");
  }
  return 0;
}
\end{verbatim}

\item Take the price and quantity of items as an input. Write a C function to calculate the sum
\end{enumerate}
of the prices. Write another C function to calculate the discount according to the
following rules:
\begin{enumerate}
\item For total less than Rs.1000, discount is 5\%.
\item For total greater than Rs.1000 but less than Rs.5000, discount is 10\%.
\item For total greater than Rs.5000, discount is 15\%.

\begin{verbatim}
#include <stdio.h>

float TotalSum(float prices[], int n) {
  float total = 0;
  for (int i = 0; i < n; i++) {
    total += prices[i];
  }
  return total;
}

float TotalDiscount(float total) {
  float discount = 0;
  if (total < 1000) {
    discount = total * 0.05;
  } else if (total < 5000) {
    discount = total * 0.10;
  } else {
    discount = total * 0.15;
  }
  return discount;
}

int main() {
  int n;
  printf("Enter the number of items: ");
  scanf("%d", &n);

  float price[n];
  printf(
         "Enter the prices of %d items(separated by spaces, then press enter): ",
         n);
  for (int i = 0; i < n; i++) {
    scanf("%f", &price[i]);
  }

  float total = TotalSum(price, n);
  printf("Total price: RS %.2f\n", total);

  float discount = TotalDiscount(total);
  printf("Discount applied: RS %.2f\n", discount);

  float finalPrice = total - discount;
  printf("Final price after discount: RS %.2f\n", finalPrice);

  return 0;
}
\end{verbatim}

\item Write a C program to accept the basic salary of an employee from the user. Calculate the gross salary on the following basis:
\end{enumerate}
Basic HRA DA
1 - 4000 10\% 50\%
4001 - 8000 20\% 60\%
8001 - 12000 25\% 70\%
12000 and above 30\% 80\%

\begin{verbatim}

#include <stdio.h>

int main() {
  float basicSalary = 0, hra = 0, da = 0, criteria = 0;
  printf("Enter the basic salary: ");
  scanf("%f", &basicSalary);

  if (basicSalary >= 1 && basicSalary <= 4000) {
    hra = basicSalary * 0.1;
    da = basicSalary * 0.5;
  } else if (basicSalary > 4000 && basicSalary <= 8000) {
    hra = basicSalary * 0.2;
    da = basicSalary * 0.6;
  } else if (basicSalary > 8000 && basicSalary <= 12000) {
    hra = basicSalary * 0.25;
    da = basicSalary * 0.7;
  } else if (basicSalary > 12000) {
    hra = basicSalary * 0.3;
    da = basicSalary * 0.8;
  } else {
    printf("Give a correct input for the basic salary");
  }

  float grossSalary = basicSalary + da + hra;

  printf("HRA: RS %.2f\n", hra);
  printf("DA: RS %.2f\n", da);
  printf("Gross Salary: RS %.2f\n", grossSalary);

  return 0;
}

\end{verbatim}


\begin{enumerate}
\item Write a C program, which will print two digit numbers whose sum of both digit is
\end{enumerate}
multiple of seven. e.g. 16,25,34\ldots{}\ldots{} 

\begin{verbatim}

#include <stdio.h>

int main() {
  for (int i = 1; i < 9; i++) {
    for (int j = 0; j < 9; j++) {
      if ((i + j) % 7 == 0) {
        printf("%d%d \n", i, j);
      }
    }
  }
  return 0;
}

\end{verbatim}

\begin{enumerate}
\item Write a C program, That reads list of n integer and print sum of product of consecutive
\end{enumerate}
numbers.if n=7 and numbers are 4,5,2,5,6,4,7
then output is 4*5+5*2+2*5+5*6+6*4+4*7 = 122.

\begin{verbatim}
#include <stdio.h>

int main() {
  int n, sum = 0, first, current_number;

  printf("Enter the number of integers in the list: ");
  scanf("%d", &n);

  printf("Enter the first integer: ");
  scanf("%d", &first);

  for (int i = 1; i < n; i++) {
    printf("Enter the next integer: ");
    scanf("%d", &current_number);
  }

  printf("Sum of products of consecutive numbers: %d\n", sum);

  return 0;
}

\end{verbatim}

\begin{enumerate}
\item Compute taxes for a given income with tax slab rates as follows\ldots{}
\end{enumerate}
slab 1\ldots{} 0-2500: 0\%
slab 2\ldots{} 2500-5000: 10\%
slab 3\ldots{} 5000-10000: 20\%
slab 4\ldots{} 10000+: 30\%
(for each range of a given income, the tax rate is different)
e.g. input: x = 5200
divisions are 0-2500, 2500-5000, 5000-5200
calculation:
0-2500 of x:
0\% of 2500 = 0
2500-5000 of x:
10\% of 2500 = 250
5000-1000 of x:
20\% of 200 = 40
output: 290

\begin{verbatim}
#include <stdio.h>

int main() {
int a, tax = 0;

printf("Enter the income: ");
scanf("%d", &a);

if (a > 10000) {
tax += (a - 10000) * 0.30;
a = 10000;
}
if (a > 5000) {
tax += (a - 5000) * 0.20;
a = 5000;
}
if (a > 2500) {
tax += (a - 2500) * 0.10;
}

printf("Tax: %d\n", tax);

return 0;
}
\end{verbatim}
\end{document}
