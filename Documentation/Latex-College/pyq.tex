\documentclass[12pt]{article}

\usepackage{amsmath} % for math symbols

\usepackage{multirow} % for multirow tables



\title{Software Tools}

\author{LastName1, FirstName1\\ \textit{your roll number}

\and LastName2, FirstName2\\ \textit{leave it as it is}}



\begin{document}



\maketitle



\section{Latex}

Prepare the question paper [1]. Every section contains marks. So construct your file as given:



\subsection{List}

\begin{itemize}

    \item orange

    \item blue

          \begin{itemize}

              \item red

                    \begin{itemize}

                        \item hello

                              \begin{enumerate}

                                  \item hello

                                  \item hello

                              \end{enumerate}

                        \item hello

                    \end{itemize}

          \end{itemize}

    \item mango

\end{itemize}



\subsection{Tables}

\begin{table}[h!]
    \centering
    \caption{Multi-row table}
    \begin{tabular}{|c|c|}
        \hline
        \textbf{first column}     & \textbf{second column} \\ \hline
        \multirow{2}{*}{Multirow} & Y                      \\ \cline{2-2}
                                  & X                      \\ \hline
        Cricket                   & Football               \\ \hline
    \end{tabular}

\end{table}



\subsection{Equations}

\begin{equation}

    \text{Precision} = \frac{\sum TP}{TP + FP}

\end{equation}



\section*{References}

\begin{itemize}

    \item[1.1] Smith, T.F., Waterman, M.S.: Identification of common molecular subsequences. J. Mol. Biol. 147, 195-197 (1981).
    \item[1.2] Smith, T.F., Waterman, M.S.: Identification of common molecular subsequences. J. Mol. Biol. 147, 195-197 (1981).

\end{itemize}



\end{document}

